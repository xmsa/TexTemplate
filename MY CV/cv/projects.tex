
\cvsection{University Projects}
\begin{cventries}
	\cventry
	{University of Isfahan} % Organisation
	{Drug-Target Interaction Prediction Using GNNs (Master's Thesis- \href{https://github.com/xmsa/Drug-target-intraction-with-gnn}{\textbf{Github Link}})} % Project
	{Isfahan, Iran} % Location
	{Sep 2022 - Sep 2025} % Date(s)
	{
		\begin{cvitems}
			\item {Developed a deep learning framework for drug–target interaction (DTI) prediction as part of the Master's thesis.}
			\item {Enhanced and integrated GNN-based models (\textbf{MHGNN-DTI}, \textbf{MHTAN-DTI}) using the \textbf{Luo dataset} from DTINet.}
			\item {Implemented preprocessing, training, validation, and deployment pipelines with \textbf{PyTorch} and \textbf{DGL}.}
			\item {\textbf{Skills:} Python, PyTorch, DGL, Graph Neural Networks, Deep Learning, Data Analysis.}
		\end{cvitems}
	}
\end{cventries}

\begin{cventries}
	\cventry
	{University of Isfahan} % Organisation
	{Stock Price Prediction using Statistical and ML Methods (Bachelors project) } % Project
	{Isfahan, Iran} % Location
	{July 2022 - Feb 2020} % Date(s)
	{
	\begin{cvitems}
		\item {Developed a stock price prediction system using statistical and machine learning methods as the Bachelor's final project.}
		\item {Applied \textbf{ARIMA} and technical indicators (MA, MACD), alongside \textbf{CNN} and \textbf{LSTM} models for time-series forecasting.}
		\item {Implemented using \textbf{Python}, with \textbf{TensorFlow}, \textbf{Django}, and \textbf{Bootstrap} for model development and deployment.}
		\item {\textbf{Skills:} Python, ARIMA, CNN, LSTM, TensorFlow, Django, Bootstrap, Time-Series Analysis.}
	\end{cvitems}
	}
	\cventry
	{University of Isfahan} % Organisation
	{Snake Game (C\#) – Github Link (\href{https://github.com/xmsa/The_Snake_game}{Basic},
		\href{https://github.com/xmsa/The-Snake-Game-Advanced}{Advanced})
	} % Project
	{Isfahan, Iran} % Location
	{Sep 2019 - Feb 2020} % Date(s)
	{
	\begin{cvitems}
		\item {Developed a \textbf{Snake game} in \textbf{C\#} as part of the “Special Topics in Computer Science” course, featuring dynamic difficulty, scoring, and game-over logic.}
		\item {Implemented core mechanics—movement, collision detection, and food generation—using \textbf{OOP} principles in \textbf{.NET/Windows Forms}.}
		\item {\textbf{Skills:} C\#, .NET, Windows Forms, Object-Oriented Programming, Problem Solving.}
	\end{cvitems}
	}
	\cventry
	{University of Isfahan} % Organisation
	{Password Manager (C\#, SQLite) - \href{https://github.com/xmsa/manpass}{\textbf{Github Link}}} % Project
	{Isfahan, Iran} % Location
	{Sep 2019 - Feb 2020} % Date(s)
	{
		\begin{cvitems}
			\item {Developed a \textbf{Password Manager} application in \textbf{C\#} using \textbf{SQLite} as part of the “Special Topics in Computer Science” course.}
			\item {Implemented secure storage and retrieval of encrypted passwords through database integration in \textbf{.NET/Windows Forms}.}
			\item {\textbf{Skills:} C\#, .NET, SQLite, Windows Forms, Data Security.}
		\end{cvitems}
	}
	\cventry
	{University of Isfahan} % Organisation
	{Daneshjooyar (C\#, SQL Server)} % Project
	{Isfahan, Iran} % Location
	{Jan 2018 - May 2018} % Date(s)
	{
		\begin{cvitems}
			\item {Designed and developed \textbf{Daneshjooyar}, a cross-platform educational application for Windows and Android as part of the “Software Design” course.}
			\item {Implemented user registration, course-based content access, and database synchronization between platforms.}
			\item {Developed the Windows version using \textbf{C\#/.NET} and \textbf{SQL Server}, and collaborated on the Android version built with \textbf{Java/Android SDK}.}
			\item {\textbf{Skills:} C\#, .NET, SQL Server, Java, Android SDK, System Analysis, Team Collaboration.}
		\end{cvitems}
	}
\end{cventries}
