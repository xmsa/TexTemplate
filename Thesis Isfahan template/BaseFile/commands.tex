% در این فایل، دستورها و تنظیمات مورد نیاز، آورده شده است.
%-------------------------------------------------------------------------------------------------------------------

% در ورژن جدید زی‌پرشین برای تایپ متن‌های ریاضی، این سه بسته، حتماً باید فراخوانی شود
\usepackage{amsthm,amssymb,amsmath}
\usepackage{tikz} % برای ترسیم نمودارهای تعاملی
\usepackage[all]{xy}
% بسته‌ای برای تنطیم حاشیه‌های بالا، پایین، چپ و راست صفحه
\usepackage[top=40mm, bottom=25mm, left=25mm, right=35mm]{geometry}
% بسته‌‌ای برای ظاهر شدن شکل‌ها و تصاویر متن


\usepackage{siunitx}
\sisetup{group-separator = {,}}

\usepackage{amssymb}
\usepackage{graphicx}
\usepackage{subcaption}
\usepackage{listings}
\usepackage{xcolor}  % برای تنظیم رنگ

\usepackage{fancyhdr}
\usepackage{fontspec}
\usepackage{framed} 
\usepackage{lastpage}
\usepackage{multirow}
\usepackage{multicol}
\usepackage{tabularx} % برای جداول با عرض متغیر
\usepackage{graphicx}
\usepackage{booktabs}
\usepackage{caption}   % برای سفارشی‌سازی کپشن‌ها
\usepackage{ifthen} % برای شرط‌ها

\captionsetup[figure]{font=small}  % کوچک کردن سایز کپشن‌ها برای تصاویر
% بسته‌ و دستوراتی برای ایجاد لینک‌های رنگی با امکان جهش
\usepackage[pagebackref=false,colorlinks,linkcolor=blue,citecolor=magenta]{hyperref}

\usepackage[nottoc]{tocbibind}
\usepackage{bidipoem}
% بسته‌ای برای رسم کادر
% بسته‌‌ای برای چاپ شدن خودکار تعداد صفحات در صفحه «معرفی پایان‌نامه»

% چنانچه قصد پرینت گرفتن نوشته خود را دارید، خط بالا را غیرفعال و  از دستور زیر استفاده کنید چون در صورت استفاده از دستور زیر‌‌، 
% لینک‌ها به رنگ سیاه ظاهر خواهند شد که برای پرینت گرفتن، مناسب‌تر است
%\usepackage[pagebackref=false]{hyperref}
% بسته‌ لازم برای تنظیم سربرگ‌ها
% بسته‌ای برای ظاهر شدن «مراجع» و «نمایه» در فهرست مطالب

% دستورات مربوط به ایجاد نمایه
\usepackage{makeidx}
\makeindex
%%%%%%%%%%%%%%%%%%%%%%%%%%
\usepackage{ptext} % بسته تولید متن ساختگی
% فراخوانی بسته زی‌پرشین و تعریف قلم فارسی و انگلیسی
\usepackage{xepersian}
\settextfont[Scale=1]{XB Niloofar}



%%%%%%%%%%%%%%%%%%%%%%%%%%
% چنانچه می‌خواهید اعداد در فرمول‌ها، انگلیسی باشد، خط زیر را غیرفعال کنید
%\setdigitfont[Scale=1.2]{XB Niloofar}
%%%%%%%%%%%%%%%%%%%%%%%%%%
% تعریف قلم‌های فارسی و انگلیسی اضافی برای استفاده در بعضی از قسمت‌های متن
\defpersianfont\nastaliq[Scale=2]{IranNastaliq}

\defpersianfont\chapternumber[Scale=3]{XB Niloofar}
\defpersianfont\naz[Scale=0.9]{IRNazanin}
\defpersianfont\nazz[Scale=1]{IRNazanin}
\defpersianfont\nazzz[Scale=1.1]{IRNazanin}
\defpersianfont\nazzzz[Scale=1.4]{IRNazanin}
\defpersianfont\nazzzzz[Scale=1.7]{IRNazanin}
\defpersianfont\Titr[Scale=1.4]{IRTitr}

\deflatinfont\Tim[Scale=.9]{Times New Roman}
\deflatinfont\Timm[Scale=1]{Times New Roman}
\deflatinfont\Timmm[Scale=1.1]{Times New Roman}
\deflatinfont\Timmmm[Scale=1.2]{Times New Roman}
\deflatinfont\Timmmmm[Scale=1.3]{Times New Roman}
\deflatinfont\Timmmmmm[Scale=1.4]{Times New Roman}

%%%%%%%%%%%%%%%%%%%%%%%%%%
% دستوری برای حذف کلمه «چکیده»
%\renewcommand{\abstractname}{}
% دستوری برای حذف کلمه «abstract»
%\renewcommand{\latinabstract}{}
% دستوری برای تغییر نام کلمه «اثبات» به «برهان»
\renewcommand\proofname{\textbf{برهان}}
% دستوری برای تغییر نام کلمه «کتاب‌نامه» به «مراجع»
%\renewcommand{\bibname}{مراجع}
\renewcommand\bibname{Reference}


\renewcommand{\indexname}{فهرست نمادها}
% دستوری برای تعریف واژه‌نامه انگلیسی به فارسی
\newcommand\persiangloss[2]{#1\dotfill\lr{#2}\\}
% دستوری برای تعریف واژه‌نامه فارسی به انگلیسی 
\newcommand\englishgloss[2]{#2\dotfill\lr{#1}\\}
% تعریف دستور جدید «\پ» برای خلاصه‌نویسی جهت نوشتن عبارت «پروژه/پایان‌نامه/رساله»
\newcommand{\پ}{پایان‌نامه }
%%%%%%%%%%%%%%%%%%%%%%%%%%


% تعریف و نحوه ظاهر شدن عنوان قضیه‌ها، تعریف‌ها، مثال‌ها و ...
\theoremstyle{definition}
\newtheorem{definition}{تعریف}[section]
\theoremstyle{theorem}
%\newtheorem{s}[definition]{}
\newtheorem{theorem}[definition]{قضیه}
\newtheorem{lemma}[definition]{لم}
\newtheorem{proposition}[definition]{گزاره}
\newtheorem{corollary}[definition]{نتیجه}
\newtheorem{remark}[definition]{تبصره}
\theoremstyle{definition}
\newtheorem{example}[definition]{مثال}
\newtheorem{notation}[definition]{نمادگذاری}
\newtheorem{definitiont}[definition]{تعریف و نمادگذاری}
\newtheorem{obs}[definition]{ملاحظه}
%%%%%%%%%%%%%%%%%%%%%%%%%%%%
\setcounter{secnumdepth}{4}


% دستورهایی برای سفارشی کردن سربرگ صفحات
\csname@twosidetrue\endcsname
\pagestyle{fancy}
\fancyhf{} 
\fancyhead[OR,ER]{\thepage}
\fancyhead[OL]{\small\rightmark}
\fancyhead[EL]{\small\leftmark}
\renewcommand{\chaptermark}[1]{%
	\markboth{\thechapter.\ #1}{}}
%%%%%%%%%%%%%%%%%%%%%%%%%%%%%



\newcommand{\rt}{\rightarrow}
\newcommand{\lrt}{\longrightarrow}
\newcommand{\lto}{\longrightarrow}
\newcommand{\st}{\stackrel}

\newcommand{\X}{\textbf{X}}
\newcommand{\Y}{\textbf{Y}}
\newcommand{\Z}{\mathbb{Z}}
\newcommand{\W}{\textbf{W}}
\newcommand{\PP}{\textbf{P}}
\newcommand{\II}{\textbf{I}}
\newcommand{\QQ}{\textbf{Q}}
\newcommand{\bb}{{\rm b}}
\newcommand{\CP}{\mathcal{P}}
\newcommand{\F}{\mathbf{F}}
\newcommand{\BF}{\mathbf{F}}
\newcommand{\CQ}{\mathcal{Q}}
\newcommand{\CD}{\mathcal{D}}
\newcommand{\CCM}{\mathcal{M}}
\newcommand{\Lf}{\mathcal{L}f}
\newcommand{\Sf}{\mathcal{S}f}

\newcommand{\Ga}{\Gamma}
\newcommand{\La}{\Lambda}
\newcommand{\om}{\omega}
\newcommand{\al}{\alpha}
\newcommand{\la}{\lambda}
\newcommand{\Tr}{{\rm{Tr}}}
\newcommand{\Dtr}{{\mathbf{D}\rm{Tr}}}
\newcommand{\DD}{{\rm{D}}}
\newcommand{\J}{\rm{J}}
\newcommand{\tc}{\bigotimes_{\mathcal{C}}}

\newcommand{\cm}{\mathbf{C}(\rm{mod} \Lambda)}
\newcommand{\cmo}{\mathbf{C}(\rm{mod} \Lambda^{\rm{op}})}
\newcommand{\cmb}{\mathbf{C}^{\rm{b}}(\rm{mod} \Lambda)}
\newcommand{\mmod}{{\rm mod}\mbox{-}}
\newcommand{\Mod}{{\rm Mod}\mbox{-}}
\newcommand{\prj}{{\rm prj}\mbox{-}}
\newcommand{\Gprj}{{\rm Gprj}\mbox{-}}
\newcommand{\Ginj}{{\rm Ginj}\mbox{-}}
\newcommand{\rad}{{\rm rad}~}
\newcommand{\inj}{{\rm inj}\mbox{-}}
\newcommand{\soc}{{\rm soc}}
\newcommand{\Tor}{{\rm Tor}}
\newcommand{\Rep}{{\rm Rep}}
\newcommand{\rep}{{\rm rep}}
\newcommand{\Inj}{{\rm Inj}\mbox{-}}
\newcommand{\GInj}{{\rm GInj}\mbox{-}}
\newcommand{\Prj}{{\rm Prj}\mbox{-}}
\newcommand{\GPrj}{{\rm GPrj}\mbox{-}}
\newcommand{\Sum}{{\rm Sum}\mbox{-}}
\newcommand{\summ}{{\rm sum}\mbox{-}}
\newcommand{\Flat}{{\rm Flat}\mbox{-}}
\newcommand{\Vecc}{{\rm Vec}}
\newcommand{\vecc}{{\rm vec}}
\newcommand{\gr}{{\rm gr}\mbox{-}}
\newcommand{\add}{{\rm add}\mbox{-}}


\newcommand{\End}{{\rm{End}}}
\newcommand{\lan}{\langle}
\newcommand{\ran}{\rangle}

\newcommand{\BC}{\mathbb{C} }
\newcommand{\BD}{\mathbf{D}}
\newcommand{\BZ}{\mathbf{Z}}
\newcommand{\D}{\mathbb{D} }
\newcommand{\K}{\mathbb{K} }
\newcommand{\CM}{{\rm CM}\mbox{-}}
\newcommand{\CF}{\mathcal{F}}
\newcommand{\CI}{\mathcal{I}}
\newcommand{\CT}{\mathcal{T}}
\newcommand{\CV}{\mathcal{V}}
\newcommand{\CS}{\mathcal{S}}
\newcommand{\CY}{\mathcal{Y}}
\newcommand{\CX}{\mathcal{X}}
\newcommand{\BI}{\mathbf{I}}


\newcommand{\N}{\mathbb{N} }
\newcommand{\Q}{\mathbf{Q} }
%\newcommand{\Z}{\mathbb{Z} }

\newcommand{\CA}{\mathcal{A} }
\newcommand{\CC}{\mathcal{C} }
\newcommand{\BX}{\mathbf{X}}
\newcommand{\BY}{\mathbf{Y}}

\newcommand{\im}{{\rm{Im}}}
\newcommand{\op}{{\rm{op}}}
\newcommand{\inc}{{\rm{inc}}}
\newcommand{\can}{{\rm{can}}}
\newcommand{\perf}{{\rm perf}}
\newcommand{\coperf}{{\rm coperf}}

\newcommand{\Coker}{{\rm{Coker}}}
\newcommand{\Ker}{{\rm{Ker}}}

\newcommand{\Hom}{{\rm{Hom}}}
\newcommand{\Ext}{{\rm{Ext}}}
%%%%%%%%%%%%%%%%%%%%%%

% دستورهایی برای سفارشی کردن صفحات اول فصل‌ها
\makeatletter
\newcommand\mycustomraggedright{%
	\if@RTL\raggedleft%
	\else\raggedright%
	\fi}
\def\@part[#1]#2{%
	\ifnum \c@secnumdepth >-2\relax
	\refstepcounter{part}%
	\addcontentsline{toc}{part}{\thepart\hspace{1em}#1}%
	\else
	\addcontentsline{toc}{part}{#1}%
	\fi
	\markboth{}{}%
	{\centering
		\interlinepenalty \@M
		\ifnum \c@secnumdepth >-2\relax
		\huge\bfseries \partname\nobreakspace\thepart
		\par
		\vskip 20\p@
		\fi
		\Huge\bfseries #2\par}%
	\@endpart}
\def\@makechapterhead#1{%
	\vspace*{-30\p@}%
	{\parindent \z@ \mycustomraggedright %\@mycustomfont
		\ifnum \c@secnumdepth >\m@ne
		\if@mainmatter
		\vskip 100\p@
		\huge\bfseries \@chapapp\space {{\chapternumber}\thechapter}
		\par\nobreak
		\vskip 40\p@
		\fi
		\fi
		\interlinepenalty\@M 
		\Huge \bfseries #1\par\nobreak
		\vskip 70\p@
}}
\makeatother

\lstdefinestyle{mypython}{
	language=Python,
	backgroundcolor=\color{lightgray},   % رنگ پس زمینه
	basicstyle=\ttfamily\footnotesize,    % فونت و اندازه
	breaklines=true,                      % خط‌شکنی خودکار
	frame=single,                         % قاب دور کد
	captionpos=b,                         % موقعیت عنوان کد (زیر)
	numbers=left,                         % شماره‌گذاری خطوط
	numberstyle=\tiny\color{gray},        % رنگ و اندازه شماره خطوط
	keywordstyle=\color{blue},            % رنگ کلمات کلیدی
	commentstyle=\color{green},           % رنگ نظرات
	stringstyle=\color{red}               % رنگ رشته‌ها
}
