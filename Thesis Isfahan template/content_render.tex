\documentclass{article}
\usepackage{amsmath}  % برای استفاده از توابع ریاضی
\usepackage{amssymb}
\usepackage{longtable}
\usepackage{amsthm}

\usepackage{array}
\usepackage{booktabs}
\usepackage{multirow}
\usepackage{multicol}
\usepackage{graphicx}
\usepackage{subcaption}
\usepackage{hyperref}
\usepackage{siunitx}
\sisetup{group-separator = {,}}
\usepackage{xepersian}
\settextfont[Scale=1]{XB Niloofar}
\newtheorem{definition}{تعریف}[section]

\begin{document}
%	\baselineskip=1cm
\section*{مقدمه}
\ptext[5-6]
\section*{نتیجه گیری}
\ptext[14-15]
\newpage
%	\baselineskip=1cm
\section*{مقدمه}
\ptext[5-6]
\section*{نتیجه گیری}
\ptext[14-15]
\newpage
%	\baselineskip=1cm
\section*{مقدمه}
\ptext[5-6]
\section*{نتیجه گیری}
\ptext[14-15]
\newpage
%	\baselineskip=1cm
\section*{مقدمه}
\ptext[5-6]
\section*{نتیجه گیری}
\ptext[14-15]
\newpage
%	\small
% شروع محیط مراجع 

\renewcommand\bibname{مراجع}
\setLTRbibitems

\bibliographystyle{IEEEtran}
\bibliography{./Refrence/myreferences.bib}
%\bibliography{./Refrence/otherreferences.bib}
		
\end{document}

