\usepackage{amsmath, amssymb, amsthm} % بسته‌های ریاضی و قضایای امسال
\usepackage[all]{xy} % برای ترسیم نمودارها
\usepackage[left=1.3cm, right=1.3cm, top=1cm, bottom=1.3cm, nohead, hmargin=2cm,vmargin=2.5cm]{geometry} % تنظیمات حاشیه‌ها
\usepackage{graphicx} % برای درج تصاویر
\usepackage{fancyhdr} % برای تنظیم سربرگ و پاورقی صفحات
\usepackage{fontspec} % برای انتخاب فونتها
\usepackage{framed} % برای ایجاد قاب برای متن
\usepackage{ } % برای شمارش تعداد صفحات
\usepackage{cite} % برای ارجاع به منابع
\usepackage{pdfpages} % برای درج صفحات PDF
\usepackage{ifthen} % برای شرطی کردن تنظیمات
\usepackage{xcolor, colortbl} % برای استفاده از رنگها
\usepackage{pifont} % برای استفاده از نمادها

\usepackage{enumitem} % برای سفارشی‌سازی محیط‌های شماره‌گذاری
\usepackage{stackengine} % برای ترتیب متون و اشیاء
\usepackage{tabularx} % برای جداول با عرض متغیر
\usepackage{multirow} % برای ترکیب سلول‌های چند ردیفی در جدول
\usepackage{fourier} % برای فونتهای تایمز
\usepackage{array} % برای سفارشی‌سازی جداول
\usepackage{makecell} % برای سفارشی‌سازی سلول‌ها در جدول
\usepackage{ptext} % تولید متن تصادفی
\usepackage{hyperref} % برای ایجاد لینکهای رنگی
\usepackage[nottoc]{tocbibind} % برای اضافه کردن فهرست مطالب به فهرست مطالب
\usepackage{bidipoem} % برای شعر فارسی
\usepackage{tikz} % برای ترسیم نمودارهای تعاملی
\usepackage{atbegshi} % برای کنترل تصاویر و نمودارها
\usetikzlibrary{calc} % کتابخانه‌ای برای ترسیم نمودارهای TikZ


\usepackage{fancyhdr} % Custom headers and footers


% چنانچه قصد پرینت گرفتن نوشته خود را دارید، خط بالا را غیرفعال و  از دستور زیر استفاده کنید چون در صورت استفاده از دستور زیر‌‌، 
% لینک‌ها به رنگ سیاه ظاهر خواهند شد که برای پرینت گرفتن، مناسب‌تر است
%\usepackage[pagebackref=false]{hyperref}


% بسته‌ لازم برای تنظیم سربرگ‌ها
% بسته‌ای برای ظاهر شدن «مراجع» و «نمایه» در فهرست مطالب

% دستورات مربوط به ایجاد نمایه
%\usepackage{makeidx}
%\makeindex
%%%%%%%%%%%%%%%%%%%%%%%%%%
% فراخوانی بسته زی‌پرشین و تعریف قلم فارسی و انگلیسی
\usepackage{xepersian}
\settextfont[Scale=1]{XB Niloofar}
%%%%%%%%%%%%%%%%%%%%%%%%%%
% چنانچه می‌خواهید اعداد در فرمول‌ها، انگلیسی باشد، خط زیر را غیرفعال کنید
\setdigitfont[Scale=1.2]{XB Niloofar}
%%%%%%%%%%%%%%%%%%%%%%%%%%


%%%%%%%%%%%%%%%%%%%%%%%%%%
% تعریف و نحوه ظاهر شدن عنوان قضیه‌ها، تعریف‌ها، مثال‌ها و ...
\theoremstyle{definition}
\newtheorem{definition}{تعریف}[section]
\theoremstyle{theorem}
%\newtheorem{s}[definition]{}
\newtheorem{theorem}[definition]{قضیه}
\newtheorem{lemma}[definition]{لم}
\newtheorem{proposition}[definition]{گزاره}
\newtheorem{corollary}[definition]{نتیجه}
\newtheorem{remark}[definition]{تبصره}
\theoremstyle{definition}
\newtheorem{example}[definition]{مثال}
\newtheorem{notation}[definition]{نمادگذاری}
\newtheorem{definitiont}[definition]{تعریف و نمادگذاری}
\newtheorem{obs}[definition]{ملاحظه}
%%%%%%%%%%%%%%%%%%%%%%%%%%%%